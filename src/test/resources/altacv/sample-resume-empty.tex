%%%%%%%%%%%%%%%%%
% This is an sample CV template created using altacv.cls
% (v1.7.2, 28 August 2024) written by LianTze Lim (liantze@gmail.com). Compiles with pdfLaTeX, XeLaTeX and LuaLaTeX.
%
%% It may be distributed and/or modified under the
%% conditions of the LaTeX Project Public License, either version 1.3
%% of this license or (at your option) any later version.
%% The latest version of this license is in
%%    http://www.latex-project.org/lppl.txt
%% and version 1.3 or later is part of all distributions of LaTeX
%% version 2003/12/01 or later.
%%%%%%%%%%%%%%%%

%% Use the "normalphoto" option if you want a normal photo instead of cropped to a circle
% \documentclass[10pt,a4paper,normalphoto]{altacv}

\documentclass[10pt,a4paper,ragged2e,withhyper,normalphoto]{altacv}
%% AltaCV uses the fontawesome5 and packages.
%% See http://texdoc.net/pkg/fontawesome5 for full list of symbols.

% Change the page layout if you need to
\geometry{left=1.25cm,right=1.25cm,top=1.5cm,bottom=1.5cm,columnsep=1.2cm}

% The paracol package lets you typeset columns of text in parallel
\usepackage{paracol}
\usepackage{hyperref}

% Change the font if you want to, depending on whether
% you're using pdflatex or xelatex/lualatex
% WHEN COMPILING WITH XELATEX PLEASE USE
% xelatex -shell-escape -output-driver="xdvipdfmx -z 0" sample.tex
\ifxetexorluatex
% If using xelatex or lualatex:
\usepackage{fontspec}
\setmainfont[
    Path = fonts/Roboto_Slab/,
    UprightFont = RobotoSlab-VariableFont_wght.ttf
]{Roboto Slab}
\setsansfont[
    Path = fonts/Lato/,
    UprightFont = Lato-Regular.ttf
]{Lato}
\renewcommand{\familydefault}{\sfdefault}
\else
% If using pdflatex:
\usepackage[rm]{roboto}
\usepackage[defaultsans]{lato}
% \usepackage{sourcesanspro}
\renewcommand{\familydefault}{\sfdefault}
\fi

% Change the colours if you want to
\definecolor{PastelRed}{HTML}{8F0D0D}
\definecolor{GoldenEarth}{HTML}{E7D192}
\definecolor{DarkPastelRed}{HTML}{450808}
\definecolor{SlateGrey}{HTML}{2E2E2E}
\definecolor{LightGrey}{HTML}{666666}

\colorlet{tagline}{PastelRed}
\colorlet{headingrule}{GoldenEarth}
\colorlet{heading}{DarkPastelRed}
\colorlet{accent}{PastelRed}
\colorlet{emphasis}{SlateGrey}
\colorlet{body}{LightGrey}

% Change some fonts, if necessary
\renewcommand{\namefont}{\Huge\rmfamily\bfseries}
\renewcommand{\personalinfofont}{\footnotesize}
\renewcommand{\cvsectionfont}{\LARGE\rmfamily\bfseries}
\renewcommand{\cvsubsectionfont}{\large\bfseries}

% Change the bullets for itemize and rating marker
% for \cvskill if you want to
\renewcommand{\cvItemMarker}{{\small\textbullet}}
\renewcommand{\cvRatingMarker}{\faCircle}
% ...and the markers for the date/location for \cvevent
\renewcommand{\cvDateMarker}{\faCalendar*[regular]}
\renewcommand{\cvLocationMarker}{\faMapMarker*}

% To display language skill fluency as string rather than stars
\newcommand{\cvskillstr}[2]{%
    \textcolor{emphasis}{\textbf{#1}}\hfill
    \textbf{\color{body}#2}\par
}

% If your CV/résumé is in a language other than English,
% then you probably want to change these so that when you
% copy-paste from the PDF or run pdftotext, the location
% and date marker icons for \cvevent will paste as correct
% translations. For example Spanish:
% \renewcommand{\locationname}{Ubicación}
% \renewcommand{\datename}{Fecha}

\begin{document}

\name{}
\tagline{}
%% You can add multiple photos on the left or right

%\photoL{2.5cm}{Yacht_High,Suitcase_High}

\personalinfo{%


  % Not all of these are required!
  %\email{your_name@email.com}
  %\phone{000-00-0000}
  %\mailaddress{Address, Street, 00000 Country}
  %\location{Location, COUNTRY}
  %\homepage{www.homepage.com}
  %\xtwitter{@twitterhandle}
  %\linkedin{your_id}
  %\github{your_id}
  %\orcid{0000-0000-0000-0000}

  %% You can add your own arbitrary detail with
  %% \printinfo{symbol}{detail}[optional hyperlink prefix]
  % \printinfo{\faPaw}{Hey ho!}[https://example.com/]
  %% Or you can declare your own field with
  %% \NewInfoField{fieldname}{symbol}[optional hyperlink prefix] and use it:
  % \NewInfoField{gitlab}{\faGitlab}[https://gitlab.com/]
  % \gitlab{your_id}
  %%
  %% For services and platforms like Mastodon where there isn't a
  %% straightforward relation between the user ID/nickname and the hyperlink,
  %% you can use \printinfo directly e.g.
  % \printinfo{\faMastodon}{@username@instace}[https://instance.url/@username]
  %% But if you absolutely want to create new dedicated info fields for
  %% such platforms, then use \NewInfoField* with a star:
  % \NewInfoField*{mastodon}{\faMastodon}
  %% then you can use \mastodon, with TWO arguments where the 2nd argument is
  %% the full hyperlink.
  % \mastodon{@username@instance}{https://instance.url/@username}
}

\makecvheader
%% Depending on your tastes, you may want to make fonts of itemize environments slightly smaller
% \AtBeginEnvironment{itemize}{\small}

%% Set the left/right column width ratio to 6:4.
\columnratio{0.6}

% Start a 2-column paracol. Both the left and right columns will automatically
% break across pages if things get too long.
\begin{paracol}{2}




\end{paracol}

\end{document}
